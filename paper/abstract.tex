\begin{abstract}
Operating systems today have numerous permission models which all aim to keep user level applications from accessing resources they shouldn't be able to. Android has a custom model built on-top of unix privledges. Unix has a basic uid/gid-based security model, to restrict applications. These basic implementations lack the necessary components to restrict applications in a meaningful manner. To address theses shortcomings operating system developers have built and included manditory access control mechanisms. SELinux is used both on Android and Linux to constrain applications from accessing or using Kernel nodes, directories, files or sockets. The BSD operating systems have two mechanisms for manditory access control. FreeBSD has Capsicum, a capability based system which essentially sandboxes applicaitons. OpenBSD has a new mechanism \textit{tame()} which is a subsystem  to restrict programs into a "reduced feature operating model". All these mechanisms have one thing in common, they're baked into the operating system and not interchangeable. We propose limiting access at the language and language runtime level, instead at the operating system level. We have implemented a proof-of-concept custom language using Racket macros, and evaulated the language on a comodoty Linux and FreeBSD machine.

\end{abstract}


