\begin{abstract}
Operating systems today have numerous permission models which all aim to keep user level applications from accessing resources they shouldn't be able to. 
Unix has a basic user and group based security model to restrict applications run by different users, but does not protect a user from his own applications.
To improve security against buggy or untrusted code, various systems have been devised to replace or augment Unix permissions.
Various mandatory access systems have been created to give more fine grained permissions to specific applications, including SELinux, Tomoyo Linux, and AppArmor on Linux, Capsicum on FreeBSD, and the \textit{tame()} mechanism on OpenBSD.
All of these systems have a complicated model, complicated configuration modifiable only by the root user, and suitable for use only by experienced system administrators rather than average users.
Android has a custom model built using Unix privledges that replaces the Unix model and limits all applications. 
However, it suffers from many problems including lack of extensibility, all-or-nothing permissions, overly-broad permissions, ineffective user interface, etc.
We propose a new model based on simple primitive permissions that allows extensibility, abstraction, run-time permission management, and configuration mechanisms that are both useful for system administrators and usable by normal users.
We have implemented a proof-of-concept custom language using Racket macros, and evaulated the language on a comodity Linux and FreeBSD machine.

\end{abstract}


